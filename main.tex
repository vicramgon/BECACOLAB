\documentclass[fleqn, 11pt, a4paper]{book}

\usepackage[utf8]{inputenc}
\usepackage{titlesec}
\usepackage[spanish]{babel}
\usepackage{afterpage}
\usepackage{graphicx}
\usepackage{fancyhdr}
\usepackage[left=2.5cm,top=2.5cm,right=2.5cm,bottom=2.5cm]{geometry}
\usepackage{float}
\usepackage{multicol}
\usepackage{array}
\usepackage{amsmath}
\usepackage{amsfonts}
\usepackage[none]{hyphenat}
\usepackage{mathtools}
\usepackage{dsfont}
\usepackage{amssymb}
\usepackage{color, colortbl}
\usepackage{pifont}
\usepackage{bm}
\usepackage{stackengine}
\usepackage{multirow}
\usepackage[table]{xcolor}
\usepackage{xcolor}
\usepackage{hhline}
\usepackage{scalerel}
\usepackage{diagbox}
\usepackage{cancel}
\usepackage[hidelinks]{hyperref}
\usepackage{listings}
\usepackage{pdfpages}
\usepackage{minitoc}
\usepackage{appendix}
\usepackage{cite}


%%%%%%%%%%%%%%%%%%%%%%%%%%%%%%%%%%%%%%%%%%%%%%%%%%%
\spanishdecimal{'}



\setlength{\headheight}{1.5cm}
\setlength{\headsep}{1cm}
\setlength{\footskip}{0cm}

\graphicspath{ {images/} }
\renewcommand\labelenumii{\theenumi.\arabic{enumii}.}
\renewcommand{\chaptername}{}

\newcommand\blankpage{%
  \null
  \thispagestyle{empty}%
  \newpage}

\pagestyle{fancy}
\fancyhf{}
\fancyhead[L]{{\scshape\large Deducción de expresiones simbólicas bajo el uso de Regresión Simbólica y Aprendizaje Profundo aplicado a Experimentos de Dinámica Newtoniana.}}
\fancyfoot[C]{ \leftmark }
\fancyfoot[R]{ \thepage }

% Abstract estilo

\newenvironment{abstract}{%
\thispagestyle{empty}
\begin{minipage}{\textwidth}
\setlength{\parskip}{0.5cm}
\setlength{\parindent}{1.5cm}
\noindent\rule{\textwidth}{1pt}

\vspace{0.2cm}

\noindent{\LARGE \textsc{Resumen}}\\
\noindent\rule{\textwidth}{1pt} 

\vspace{1cm}

}
{\end{minipage}
\newpage}

\newenvironment{acknowledgments}{%
\thispagestyle{empty}
\begin{minipage}{\textwidth}
\setlength{\parskip}{0.5cm}
\setlength{\parindent}{1.5cm}
\noindent\rule{\textwidth}{1pt}

\vspace{0.2cm}

\noindent{\LARGE \textsc{Agradecimientos}}\\
\noindent\rule{\textwidth}{1pt} 

\vspace{1cm}

}
{\end{minipage}
\newpage}

%=======================================
\includeonly
{
	sections/portada,
	sections/resumen,
	sections/cap1,
	sections/cap2,
	sections/apd1,
}

\sloppy
\begin{document}

\setlength{\parskip}{0.5cm}

\renewcommand{\listtablename}{Índice de tablas}
\renewcommand{\tablename}{Tabla}

\renewcommand{\lstlistlistingname}{Índice de fragmentos de código}
\renewcommand{\lstlistingname}{Código}
\renewcommand{\mtctitle}{}

\dominitoc

\includepdf[scale=1]{sections/portada2.pdf}

\blankpage

\begin{titlepage}
\begin{center}
\includegraphics[scale=0.12]{logoETSIIUSLSI.png}

\vspace{1cm}

{\large \textsc{Universidad de Sevilla \\ Escuela Técnica Superior de Ingeniería Informática.}}

\vspace{0.5cm}

{\large \textsc{G.I.I. Tecnologías Informáticas.}}

\vspace{1cm}

{\Large \textsc{Proyecto de Beca de Colaboración}}

\vspace{1cm}


{\LARGE Deducción de expresiones simbólicas bajo el uso de Regresión Simbólica y Aprendizaje Profundo (DL) aplicado a Experimentos de Dinámica Newtoniana.}

\vspace{1cm}

{\large \textsc{Realizado por:}}

{\large Ramos González, Víctor}

\vspace{0.5cm}

{\large \textsc{Dirigido por:}}

{\large Álvarez García, Juan Antonio}\\
{\large Sancho Caparrini, Fernando}

\vspace{0.5cm}

{\large \textsc{En colaboración con el departamento de}}

{\large Lenguajes y Sistemas Informáticos.}
\end{center}

\vspace{1cm}
\begin{flushright}
\textit{Sevilla a ... de ... del 2021 \quad}
\end{flushright}
\end{titlepage}


\blankpage

\begin{abstract}

TO DO...

\end{abstract}

\blankpage


%\include{sections/agradecimientos}

%\blankpage

\setcounter{page}{1}%

\renewcommand{\contentsname}{Índice de contenidos}

\pagestyle{empty}

\tableofcontents

\addtocontents{toc}{\textbf{Contenidos}\hfill \textbf{Página} \par}
\addtocontents{toc}{\vspace{-2mm} \hspace{-7.5mm} \hrule \par}

\newpage
%\listoftables

%\newpage
%\lstlistoflistings


\chapter{Introducción.}
\minitoc

\newpage

\pagestyle{fancy}

\include{sections/cap1}

\chapter{Fundamentos Teóricos.}

\minitoc

\newpage

\section{Visión general del capítulo.}

\section{Redes Neuronales Gráficas (GNN).}

\section{Regresión Simbólica.}

\section{...}



\appendix

\chapter{Primer Apéndice}

\minitoc

\newpage 

\section{Visión general del apéndice}
\section{...}

Puedes verlo en \cite{Patricio2011}. Te recomiendo leer \cite{Patricio2011, Zacarias2009, Alfonso2010b, Alfonso2010a}.

\renewcommand\bibname{Referencias}
\bibliographystyle{acm}
\bibliography{referencias}



\end{document}






